% ==============================================================================
% DVG001
% Introduktion till Linux och små nätverk
% Inlämn
% ---------------------------------------
% Author:
% Jonas Sjöberg     <tel12jsg@student.hig.se>
%
% License:
% Creative Commons Attribution-NonCommercial-ShareAlike 4.0 International
% See LICENSE.md for full licensing information.
% ==============================================================================

\section{Inledning}\label{inledning}
% Skriv en kort inledning här som beskriver kortfattat vad rapporten handlar
% om. Den skall vara orienterande om Bakgrund och Syfte.
Rapporten beskriver installationen av ett operativsystem i en
virtuell maskin. Operativsystemet kommer att utgöra den arbetsmiljö som används
under kursen. Det operativsystem som valts är \texttt{Debian} som är en
distribution av \texttt{Linux}.


% ~~~~~~~~~~~~~~~~~~~~~~~~~~~~~~~~~~~~~~~~~~~~~~~~~~~~~~~~~~~~~~~~~~~~~~~~~~~~~~
\subsection{Bakgrund}
% Beskriv lite mer ingående om bakgrunden till uppgiften, vad den handlar om.
Den första laborationen i kursen ''DVG001 --- Introduktion till Linux och små
nätverk'' beskriver installation och konfigurering av den arbetsmiljö som
kommer att användas under kursens gång. Operativsystemet \texttt{Linux } ska
installeras på en dator av valfri typ.  Datorn kan även vara helt virtuell och
körs då parallelt och under ett värdsystem. Rapporten beskriver hur
operativsystemet installeras i den virtuella maskinen.


% ~~~~~~~~~~~~~~~~~~~~~~~~~~~~~~~~~~~~~~~~~~~~~~~~~~~~~~~~~~~~~~~~~~~~~~~~~~~~~~
\subsection{Syfte}
% Skriv lite mer ingående om syftet med uppgiften.
Syftet med laborationen är att konstruera en arbetsmijö som kan användas under
resten av kursen. Att installera ett operativsystem ger en god insyn i hur 
systemet är uppbyggt ''under huven'' och kan vara mycket upplysande.
Många gånger ger oförutsedda fel möjligheter till fördjupning och
efterforskning. Det är lämpligt att kursen börjar med denna övning.


% ~~~~~~~~~~~~~~~~~~~~~~~~~~~~~~~~~~~~~~~~~~~~~~~~~~~~~~~~~~~~~~~~~~~~~~~~~~~~~~
\subsection{Nomenklatur}
I texten används \emph{värdsystem} (från engelskans \emph{hostname}) för att
referera till den fysiska dator som kör programmet \texttt{VirtualBox}.
Värdsystemet har direkt kontroll till all fysisk hårdvara.
\par Den virtuella maskinen som körs i \texttt{VirtualBox} kommer i texten att
kallas \emph{gästsystem} (efter engelskans \emph{guest}).
