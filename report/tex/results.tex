% ==============================================================================
% DVG001
% Introduktion till Linux och små nätverk
% ---------------------------------------
% Author:
% Jonas Sjöberg     <tel12jsg@student.hig.se>
%
% License:
% Creative Commons Attribution-NonCommercial-ShareAlike 4.0 International
% See LICENSE.md for full licensing information.
% ==============================================================================


\section{Resultat}
% Vad är det konkreta resultatet av uppgiften?
Det konkreta resultatet av uppgiften är att vi har skapat en arbetsmiljö för
användning under kursens gång. Arbetsmiljön är i form av en nyskapad virtuell
maskin installation av Debian 8.  Arbetsmiljön ger ett säkert sätt kan
experimentera och utföra kommande laborationer i en omgivning som är isolerad
från omgivningen på ett sätt som ger säkerhet och möjlighet att jobba med en
''ren'' installation, Dessutom ger möjligheter för att spara läget en virtuell
maskin befinner sig i genom \texttt{Snapshots} i \texttt{VirtualBox} en extra
säkerhet. Om ett fel skulle uppstå finns alltid möjligheten att återställa
systemet till ett känt fungerande läge, vilket sparar enormt mycket tid.


\section{Diskussion}
% Diskutera lite friare om vad resultatet betyder samt vad ni mer lärt er.
% Även om hur ni kanske skulle gjort annorlunda om ni gjort om det.
En helt virtualiserad miljö ger bättre säkerhet och feltolerans då den är
isolerad från värdsystemet. Det finns således ingen risk att känsliga data
raderas av misstag eller att ett allvarligt systemfel äventyrar funktionalitet
i det övriga systemet. Med riktiga inställningar för fildelning genom nätverk
kan det isolerade systemet hållas isolerat till den grad att även ett totalt
systemhaveri inte ''läcker ut'' till värdsystemet. T.ex. vid test av
''malware'' så är detta är en mycket viktig aspekt ur säkerhetssynpunkt.
\par En stor del av arbete med mjukvarusystem består av att läsa dokumentation
och på egen hand lösa problem efter att de dyker upp. När ett oförutsett fel
uppstår tenderar jag allt oftare att som första steg läsa loggfiler och själv
försöka felsöka snarare än att använda Google som första utväg. För att uppnå
en verkligt hög nivå av förståelse (''grokka'') något ämne så tror jag att
det är mycket viktigt att ha en känsla för äventyr och experimentation, att
själv försöka lösa problem och möta behov som uppstår.


\section{Slutsatser}
% Sammanfatta vad ni fått för resultat, utgå från syftet som ni angett i
% Syfte ovan.
Vi har uppnått det mål vi satte upp, dvs att konstruera en arbetsmijö som kan
användas under resten av kursen. Under installationsprocessen uppstod några
problem som vi tvingades lösa och problemlösning kan anses vara ett bra sätt
att få god insyn och djupare förståelse för hur olika system fungerar ''under
huven''. Avslutningsvis kan sägas att laborationer av den här typen uppskattas
då de är intressanta och många gånger lärorika.
