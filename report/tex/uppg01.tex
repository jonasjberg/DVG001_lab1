% ==============================================================================
% DVG001
% Introduktion till Linux och små nätverk
% ---------------------------------------
% Author:
% Jonas Sjöberg     <tel12jsg@student.hig.se>
%
% License:
% Creative Commons Attribution-NonCommercial-ShareAlike 4.0 International
% See LICENSE.md for full licensing information.
% ==============================================================================

\section{Planering}
% TODO: Kort och orienterande om hur ni tänkte genomföra uppgiften.
%       Orienterande om Planering och Genomförande.


% ~~~~~~~~~~~~~~~~~~~~~~~~~~~~~~~~~~~~~~~~~~~~~~~~~~~~~~~~~~~~~~~~~~~~~~~~~~~~~~
\subsection{Arbetsmetod}
% Hur kommer ni att arbeta?  Detta är en lite längre text än den rent
% orienterande texten i Planering och genomförande ovan.

Nedan följer en preliminär redogörelse för den experimentuppställning som används
under laborationen:

\begin{itemize}
  \item Laborationen utförs på en \texttt{ProBook-6545b} laptop som kör
        \texttt{Xubuntu 15.10} på \texttt{Linux 3.19.0-28-generic}.

  \item Rapporten skrivs i \LaTeX\ med texteditorn \texttt{Vim} och kompileras
        till pdf med \texttt{latexmk}.

  \item För revisionskontroll används \texttt{Git}.

  \item Virtualisering sker med \texttt{Oracle VirtualBox} version
        \texttt{5.0.10\_Ubuntu r104061}.

\end{itemize}


% ~~~~~~~~~~~~~~~~~~~~~~~~~~~~~~~~~~~~~~~~~~~~~~~~~~~~~~~~~~~~~~~~~~~~~~~~~~~~~~
\section{Genomförande}
% Här skriver ni vilka steg ni gjorde och resultatet av dem. Ni skall ha med
% information så att vi kan se hur ni har gjort, dvs beskrivande text,
% skärmdumpar, bilder etc.  Längre skärmdumpar, innehåll i relevanta filer och
% större bilder lägger ni i bilagor, som bilaga I, så att de inte tar över en
% sida själva.
% Kommandon som ni skriver i ett skal skall skrivas i detta format, som är
% teckenformatmall ”Exempel” i OpenOffice/LibreOffice. Detta så att de
% skiljer sig från övriga brödtext i stycket.  Detta underlättar läsningen för
% andra, som oss lärare.


% ______________________________________________________________________________
\subsection{Att skapa en virtuell maskin}
Det första steget var att ladda hem en bootbar \texttt{.iso}-fil av
distributionen Debian. Det finns många olika varianter att välja bland och då
jag sedan tidigare är van vid att använda \texttt{xfce} så valde jag att ladda
hem \texttt{debian-live-8.2.0-i386-xfce-desktop} som är en variant av Debian
som levereras med den grafiska miljön \texttt{xfce}. Den Grafiska miljön är
bara en komponent av hela systemet som lätt kan bytas ut vid ett senare
tillfälle, så att välja precis ``rätt'' paktererad version av en distribution
är inte nödvändigt.  Men en färdigpaketerad version brukar kunna förenkla
installationsprocessen.

\screenshot{img/A_new-01}
           {Värdsystemet}
           {Skärmdump av värdsystemet som kör \texttt{VirtualBox} där en ny
            virtuell maskin skapas genom att klicka den blå ikonen
            \texttt{New}.}
           {fig:01}

Jag väljer att skapa min miljö som en virtuell maskin. Då jag redan använder
och är bekant med \texttt{VirtualBox} så väljer jag att använda det för att
skapa min virtuella maskin enligt Figur~\ref{fig:01}.

\screenshot{img/A_new-02}
           {Ny virtuell maskin}
           {I dialogrutan för en ny virtuell maskin väljs namnet
            \texttt{debian}. Autodetektering baserat på det inmatade
            namnet markerar automatiskt \texttt{Type} samt \texttt{Version}
            som i det här fallet stämmer och är lämpliga inställningar.}
           {}

\screenshot{img/A_new-03}
           {Allokering av RAM-minne}
           {Storleken på RAM-minnet sätts till en lämplig kompromiss mellan
            vad värdsystemet kan tänkas behöva och vad gästsystemet kräver.}
           {}

\screenshot{img/A_new-04}
           {Skapa ny virtuell hårddisk}
           {I dialogrutan för hårddisk väljer vi att skapa en ny virtuell
            hårddisk.}
           {}

\screenshot{img/A_new-05}
           {Typ av virtuell lagringsmedia}
           {Standardvalet \texttt{VDI} fungerar bra för det här
            användningsområdet. Ytterigare information om de olika alternativen
            har hämtats från dokumentationen till \texttt{Virtualbox},
            se \cite{virtualbox:vdidetails}.}
           {}

\screenshot{img/A_new-06}
           {Dynamiskt lagringsutrymme}
           {Lagringsutrymmets storlek väljs att växa dynamiskt efter behov.}
           {}

\screenshot{img/A_new-07}
           {Namnge virtuell lagringsmedia}
           {Namnet och storleken på disken sätts till \texttt{debian} (vilket
            ger ett filnamn \texttt{debian.vdi} i katalogen för den virtuella
            maskinen) och storleken sätts godtyckligt till $\SI{10}{\mega\byte}$.}
           {}

\screenshot{img/A_new-08}
           {Fortsatt konfiguration}
           {Ytterligare inställningsmöjligheter kan göras genom att högerklicka
            på den virtuella maskinen vi skapat och välja \texttt{Settings}.
            Under fliken \texttt{Basic} så ser vi att \texttt{Name}, \texttt{Type}
            och \texttt{Version} är valda sedan tidigare.}
           {}

\screenshot{img/A_new-09}
           {Stöd för delad ``clipboard''}
           {Under fliken \texttt{Advanced} aktiveras stöd för delad ``clipboard'',
            vilket gör det möjligt att kopiera och klistra in text mellan
            värdsystemet och den virtuella maskinen.}
           {}

\screenshot{img/A_new-10}
           {Grafikinställningar}
           {Bland alternativen till höger väljs \texttt{Display}.
            Grafikminnet för gästen ökas till värdsystemets högsta möjliga
            $\SI{128}{\mega\byte}$. Stöd för hårdvaruacceleration finns ej med
            den här kombinationen av grafikkort och version av \texttt{Xorg}.}
           {}

\screenshot{img/A_new-11}
           {Inställningar för lagringsenheter}
           {Under alternativet \texttt{Storage} finns inställningar för den
            virtuella maskinens lagringsenheter. Under \texttt{Controller: IDE}
            finns en emulerad optisk enhet tillgänglig.
            Och under \texttt{Controller: SATA} syns den virtuella hårddisken
            \texttt{debian.vdi}.}
           {}

\screenshot{img/A_new-12}
           {Montera diskavbildningsfil}
           {Diskavbildningsfilen \texttt{debian-live-8.2.0-i386-xfce-desktop} monteras
            genom att klicka på den lilla ikonen av en CD-skiva till höger om
            \texttt{Optical Drive: IDE Secondary Master}. Om kryssrutan
            \texttt{Live CD/DVD} markeras så matas det virtuella mediet ut vid omstart
            efter att installationen är slutförd. Utmatning av media sköts manuellt.}
           {}

\screenshot{img/A_new-13}
           {Inställningar för fildelning}
           {Under \texttt{Shared Folders} väljs en katalog för fildelning mellan den
            virtuella maskinen och värdsystemet. Katalogen är gemensam för alla
            virtualiserade system på värdsystemet. \texttt{Auto-mount} markeras vilket
            aktiverar automatisk montering av katalogen. Det är ett exempel på
            bekvämlighetsfunktioner som följer med \texttt{Guest Additions}.}
           {}

\screenshot{img/A_new-14}
           {Inställningar för fildelning}
           {Här syns den delade mappen \texttt{VirtualBoxShared} som nyss lagts till.}
           {}

% ______________________________________________________________________________
\subsection{Installation av operativsystemet}
Den nya virtuella maskinen har nu konfigurerats för att fungera
tillfredsställande på värdsystemet och är redo för installation av
operativsystemet.

\screenshot{img/B_install-01}
           {Första boot av systemet}
           {Systemet bootas för första gången och följande skärm visas efter att
            bios laddat klart och läst in den bootbara diskavbildningsfilen
            \texttt{debian-live-8.2.0-i386-xfce-desktop}.}
           {}

\screenshot{img/B_install-02}
           {Påbörja installationsprocessen}
           {Installationen startas genom att välja antingen \texttt{Install} eller
            \texttt{Graphical Install}. I det här fallet väljs det senare.}
           {}

\screenshot{img/B_install-03}
           {Val av språk}
           {Språket sätts till engelska. Valet motiveras dels av att
            översättningen i många fall inte är komplett och att systemet
            känns mer koherent på engelska, men i huvudsak så är
            det en smaksak.}
           {}

\screenshot{img/B_install-04}
           {Val av geografisk plats}
           {Vid valet för geografisk plats väljs \texttt{other}.}
           {}

\screenshot{img/B_install-05}
           {Vidare val av plats -- \texttt{locale}}
           {Vidare väljs \texttt{Europe}.}
           {}

\screenshot{img/B_install-06}
           {Ytterligare val av plats -- \texttt{locale}}
           {Och slutligen \texttt{Sweden}.}
           {}

\screenshot{img/B_install-07}
           {Val av \texttt{locale}}
           {För locale väljs \texttt{en\_US.UTF-8}. Egentligen vore det möjligtvis
            bättre lämpat att välja en svensk locale men tidigare erfarenheter har
            cementerat vanan att välja en engelsk locale, och om nödvändigt
            ändra till en svensk senare, när systemet är installerat.}
           {}

\screenshot{img/B_install-08}
           {Val av tangentbordslayout}
           {Tangentbordslayouten ställs till svensk. Valet av
            tangentbordslayout är skiljd från det tidigare valet av
            \texttt{locale}, det är alltså möjligt att använda ett svenskt
            tangentbord men låta systemet skriva ut valuta, tidszon, etc. efter
            amerikanska standarder.}
           {}

\screenshot{img/B_install-09}
           {Installationsprocessen fortgår}
           {I det här skedet används den information som matats in för att
            påbörja installationen. Ingen interaktion krävs på ett tag.}
           {}

\screenshot{img/B_install-10}
           {Val av \texttt{hostname}}
           {Datorns namn på nätverket bestäms av \texttt{hostname} som sätts till
            ''debian''. Det gör det möjligt att ansluta till datorn via t.ex.
            \texttt{SSH} senare genom att skriva ''\texttt{ssh jonas@debian}''.}
           {}

\screenshot{img/B_install-11}
           {Konfigurering av nätverk}
           {Domännamnet lämnas tomt.}
           {}

\screenshot{img/B_install-12}
           {Lösenord till användaren \texttt{root}}
           {Administratörskontot \texttt{root} får ett nytt lösenord.
            Lösenordet matas in två gånger för att verifiera inmatningen.}
           {}

\screenshot{img/B_install-13}
           {Skapandet av en ny användare}
           {Ett nytt användarkonto skapas och ett riktigt för- och efternamn
            skrivs in.  Uppgifterna används av olika program, bl.a.
            \texttt{OpenOffice}.}
           {}

\screenshot{img/B_install-14}
           {Val av den nya användarens användarnamn}
           {Den nya användarens användarnamn väljs enligt god praxis
            \cite{forelasning03} upp till 8 små bokstäver/siffror (börjar
            med \texttt{a-z} och sedan upp till sju ''\texttt{0-9a-z\_-}''.)}
           {}

\screenshot{img/B_install-15}
           {Val av den nya användarens lösenord}
           {Den nya användaren \texttt{jonas} får ett nytt lösenord.}
           {}

\screenshot{img/B_install-16}
           {Partiotionering av hårddisken}
           {I det här skedet börjar partitioneringen av hårddisken, en
            uppdelning av det fysiska lagringsutrymmet till logiska
            partitioner. I vanliga fall är det lämpligt att skapa flera
            partitioner och separera användarnas filer från systemets filer,
            men i det här fallet väljs den automatiska metoden
            \texttt{Guided - use entire disk} där hela disken används.}
           {}

\screenshot{img/B_install-17}
           {Bekräftelse av partitionering}
           {En extra dialogruta informerar om att disken kommer att raderas.}
           {}

\screenshot{img/B_install-18}
           {Val av partitionsschema}
           {Disken \texttt{sda} är vald för partitionering och vi väljer här
            den enkla metoden där hela disken används för en enda stor partition.
            Det är som sagt ofta lämpligt att välja ett partitionssysten där
            åtminstone \texttt{/home} är förlagt till en egen partition, men
            eftersom att detta system är ett isolerat system som bara kommer
            att användas som arbetsmiljö för kursen duger det enklaste alternativet,
            \texttt{All files in one partition (recommended for new users)}.
            }
           {}

\screenshot{img/B_install-19}
           {Bekräftelse av partitionering}
           {Ytterligare en dialogruta visar de valda inställningarna och ger
            chans att avbryta innan all data rensas och partitioneringen utförs.}
           {}

\screenshot{img/B_install-20}
           {Start av partitionering}
           {Partitionering startar genom att klicka \texttt{Continue}.}
           {}

\screenshot{img/B_install-21}
           {Systemet installeras}
           {Installationen sköter sig själv ett tag, filer laddas hem, installeras
            och konfigureras med de inställningar som hittils gjorts.}
           {}

\screenshot{img/B_install-22}
           {Val av nätverksspegel}
           {Då en stadig uppkoppling finns tillgänglig så används en
           nätverksspegel (engelska \texttt{network mirror}) för att ladda hem
           nyare versioner av paket under installationsprocessen.}
           {}

\screenshot{img/B_install-23}
           {Val av nätverksspegelns server}
           {Servern väljs efter geografiskt läge för att säkerställa god
            uppkoppling med snabb överföringshastighet.}
           {}

\screenshot{img/B_install-24}
           {Val av \texttt{Debian archive mirror}}
           {Här väljs en specifik server som kommer att användas för att
            ladda ned paket.}
           {}

\screenshot{img/B_install-25}
           {Val av proxy för nätverksspegel.}
           {Fältet för proxy lämnas tomt.}
           {}

\screenshot{img/B_install-26}
           {Installations-/filöverföringsprocess}
           {Återigen så dröjer det en stund medan systemet laddar hem,
            installerar och Konfigurerar paket.}
           {}

\screenshot{img/B_install-27}
           {Installation av \texttt{GRUB}}
           {Intallationen av \texttt{GRUB} startar i det här läget.}
           {}

\screenshot{img/B_install-28}
           {Inställningar för \texttt{GRUB}}
           {Eftersom det bara finns ett operativsystem på den här hårddisken så
            skrivs \texttt{GRUB} \texttt{boot loader} direkt till
            \texttt{master boot record}.
            Det innebär att då systemet bootar och \texttt{BIOS} laddat klart
            så letar det på en särskild plats i början av disken efter något
            att exekvera, där kommer nu \texttt{GRUB} att ta över och fortsätta
            start av operativsystemet.}
           {}

\screenshot{img/B_install-29}
           {Val av \texttt{boot loader}}
           {Här väljs att \texttt{GRUB} \texttt{boot loader} ska installeras
            på vår enda tillgängliga partition.}
           {}

\screenshot{img/B_install-30}
           {Installationen slutförd}
           {Installationsprocessen är nu klar och vi ombeds mata ut den media
            vi använt för att boota till för att sedan starta om datorn.}
           {}

\screenshot{img/B_install-31}
           {Slutförande och uppstädning}
           {Efter att installationsprogrammet rensat upp eventuella tillfälliga filer
            och skrivit de sista ändringarna till hårddisken startar systemet om.
            \par Här stöter vi dock på ett allvarligt problem.}
           {}


% ______________________________________________________________________________
\subsection{Felsökning}
Direkt efter den första omstarten så uppstod ett allvarligt fel. Den virtuella
maskinen havererade totalt och stängdes ned. Lyckligtvis så var det inte alltför
svårt att lösa problemet och komma vidare.

\screenshot{img/C_debug-01}
           {Kritiskt fel hos \texttt{VirtualBox}}
           {Vid den första omstarten som sker efter att installationsprogrammet
            avslutats så kraschar den virtuella maskinen helt och ovanstående
            felmeddelande visas.}
           {}

\screenshot{img/C_debug-02}
           {Detaljerad felbeskrivning}
           {Under \texttt{Details} visas ytterligare information om felet:
             \texttt{The guest is trying to switch to the PAE mode which is
            currently disabled by default in VirtualBox ..}
            Det här är ett problem som är relativt enkelt att lösa.}
           {}

\screenshot{img/C_debug-03}
           {Korrigera inställningar för \texttt{PAE/NX}}
           {För att lösa problemet måste den virtuella maskinens inställningar
            ändras. Det görs genom att välja \texttt{Settings} och vidare
            \texttt{System}, varvid kryssrutan för \texttt{Enable PAE/NX}
            aktiveras.  Systemet startas sedan om.}
           {}


% ______________________________________________________________________________
\subsection{Första start av systemet}

\screenshot{img/D_first-boot-01}
           {En första lyckad systemstart}
           {Efter att inställningarna ändrats så visas inte felmeddelandet och
            systemet startar utan problem. En grafisk inloggningsskärm startar
            automatiskt.}
           {}

\screenshot{img/D_first-boot-02}
           {Den första inloggningen}
           {Användaren \texttt{jonas}, som skapades under
            installationsprocessen, används för att logga in.  Lösenordet är
            det som angavs under installationen.}
           {}

\screenshot{img/D_first-boot-03}
           {Start av det grafiska gränssnittet \texttt{xfce}}
           {Efter att ett korrekt användarnamn och lösenord matats in startas
            \texttt{xfce4}, som är den grafiska miljö
            (\texttt{desktop environment}) som är standard i den distribution
            som använts. Andra versioner av \texttt{Debian} kommer med andra
            grafiska miljöer. Den övre panelen visar en dialogruta som frågar
            om standardinställnigarna ska användas. Här väljs att använda
            standard, \texttt{use default config}.}
           {}


% ______________________________________________________________________________
\subsection{Installation av \texttt{Guest additions}}

\screenshot{img/E_guest-install-01}
           {Börja installationen av \texttt{Guest additions}}
           {\texttt{VirtualBox} \texttt{Guest additions} är ett tillägg som
            aktiverar extra funktionalitet och bekvämlighetsfunktioner. T.ex.
            enklare fildelning mellan gäst- och värdsystem, samt bättre
            möjligheter att ställa in skärmupplösning. För att installera
            \texttt{Guest additions} används menyalternativet
            \texttt{Insert Guest Additions CD image...} i den övre menyn i
            \texttt{VirtualBox}. Precis som vid installationen av operativsystemet
            så används en diskavbildningsfil som emulerar en CD/DVD-skiva.}
           {}

\screenshot{img/E_guest-install-02}
           {Automatisk nedladdning av diskavbildningsfilen}
           {Tidigare behövde man ladda hem \texttt{Guest additions} separat från
            Oracles hemsida. I senare versioner av \texttt{VirtualBox} finns
            möjligheten att automatiskt ladda hem de filer som saknas.
            \par Av bekvämlighetsskäl låter vi programmet sköta nedladdningen
            automatiskt varpå alternativet \texttt{Download} väljs.}
           {}

\screenshot{img/E_guest-install-03}
           {Bekräftelse av nedladdning}
           {En extra bekräftelse krävs för att nedladdningen ska börja.}
           {}

\screenshot{img/E_guest-install-04}
           {Okänt fel}
           {Efter att nedladdningen till slut nått 100\% uppstår ett fel:
            \par \texttt{The network operation failed with the following error:
            \\ During network request: Unknown reason.}.
            \par Felmeddelandet är inte särskilt hjälpsamt vilket gör det
            svårare att veta vart och hur det är bäst lämpat att börja felsöka.}
           {}


% ______________________________________________________________________________
\subsection{Felsökning av \texttt{Guest additions}}

\screenshot{img/F_guest-debug-01}
           {Sökning i paketarkiven på värdsystemet}
           {Eftersom att nedladdningen av \texttt{Guest additions} sker på
            värdsystmet börjar felsökningen med att söka efter filerna i
            paketarkiven. Kommandot \texttt{apt-cache search} används för att
            söka efter ''virtualbox''. Sökresultaten skickas sen till
            \texttt{grep} som vidare söker efter ordet ''guest''.
            \par Bland resultaten finns \texttt{virtualbox-guest-additions-iso}
            vilket troligtvis är den diskavbildningsfil som behövs.}
           {}

\screenshot{img/F_guest-debug-02}
           {Manuell installation av \texttt{virtualbox-guest-additions-iso}}
           {\texttt{Guest additions} installeras manuellt med kommandot
            \texttt{apt-get install}, som tar de paket som ska installeras som
            argument. I det här fallet \texttt{virtualbox-guest-additions-iso}.
            Kommandot kräver administratörsrättigheter och körs därför med
            \texttt{sudo}.}
           {}

\screenshot{img/F_guest-debug-03}
           {Lokalisering av diskavbildningsfilen}
           {Efter att installationen slutförts listas innehållet i det
            installerade paketet genom att skriva kommandot 
            \texttt{dpkg -L <paketnamn>}.  Alla filer listas med fullständig
            sökväg. Filen som ska monteras i \texttt{VirtualBox} har den
            fullständig sökvägen
            \texttt{/usr/share/virtualbox/VBoxGuestAdditions.iso}.}
           {}

\screenshot{img/F_guest-debug-04}
           {Montering av diskavbildningsfilen}
           {Diskavbildningsfilen monteras nu som tidigare genom menyn
            \texttt{Devices} och sedan \texttt{Optical Drives}, sedan väljs
            filen som lokaliserades i föregående steg.}
           {}

\screenshot{img/F_guest-debug-05}
           {Montering av virtuell media}
           {Diskavbildningsfilen dyker nu upp som en transparent ikon på
            skrivbordet i gästsystemet. För att komma åt filerna väljs
            \texttt{Mount Volume} i menyn som dyker upp efter att ikonen
            högerklickats.}
           {}

\screenshot{img/F_guest-debug-06}
           {Visa media i filhanteraren}
           {Ikonen som tidigare var transparent är nu helt ifylld, vilket
            innebär att den är monterad och tillgänglig. Genom att högerklicka
            och välja \texttt{Open} så öppnas platsen där diskavbildningsfilen
            monterats i filhanteringsprogrammet \texttt{Thunar}, som är 
            standard i \texttt{xfce}.}
           {}

\screenshot{img/F_guest-debug-07}
           {Media öppen i \texttt{Thunar}}
           {Här visas diskavbildningsfilen i kolumnen till vänster under
            \texttt{DEVICES} som \texttt{VBOXADDITI...}. I adressfältet visas
            den fullständiga sökvägen där diskavbildningsfilen monterats:
            \texttt{/media/cdrom0/}. Ett terminalfönster öppnas genom att
            högerklicka i fönstret och välja \texttt{Open Terminal Here} i
            menyn som dyker upp.}
           {}

\screenshot{img/F_guest-debug-08}
           {Listning av alla filer i diskavbildningsfilen}
           {I terminalfönstret körs kommandot \texttt{ls -ahl} för att visa
            alla filer i katalogen.}
           {}

\screenshot{img/F_guest-debug-09}
           {Installation av \texttt{Guest Additions}}
           {För att installera \texttt{Guest Additions} krävs
            administratörrättigheter. För att logga in som en annan användare
            används kommandot \texttt{su}. Om det körs utan extra argument 
            byter man till användaren \texttt{root}, förutsatt att lösenordet
            till kontot \texttt{root} matas in. Efter att ha bytt till
            administratörskontot startas installtationen genom att skriva
            kommandot \texttt{sh VBoxLinuxAdditions.run}, som startar ett nytt
            skal och skickar installationsfilen \texttt{VBoxLinuxAdditions.run}
            som argument till skalet som till följd exekverar installationsfilen.}
           {}

\screenshot{img/F_guest-debug-10}
           {Slutförd installation av \texttt{Guest Additions}}
           {Efter en stund har installationen av \texttt{Guest Additions}
            slutförts och vi ombeds starta om fönsterhanteraren/systemet.
            Den virtuella maskinen startas om genom att välja restart i 
            den övre menyraden.}
           {}


% ______________________________________________________________________________
\subsection{Slutförd installation}

\screenshot{img/G_done-01}
           {Omstart efter slutförd installation}
           {Efter att systemet startat om sker åter igen inloggning med
            användaren \texttt{jonas}.}
           {}

\screenshot{img/G_done-02}
           {Aktivering av helskärmsläge}
           {Med \texttt{Guest Additions} installerat kan den virtuella maskinen
            köras i helskärmsläge genom att välja \texttt{Full-screen Mode} i
            menyn \texttt{View}.}
           {}

\screenshot{img/G_done-03}
           {Slutresultatet}
           {Bilden visar det slutgiltiga resultatet. Den virtuella maskinen
            körs i helskärm och fungerar precis som om systemet kördes
            ''direkt mot metallen'', som värdsystemet.
            Med \texttt{Guest Additions} installerat justeras fönstret och
            skärmupplösningen automatiskt för att passa värdsystemet.}
           {}

