% ==============================================================================
% DVG001
% Introduktion till Linux och små nätverk
% ---------------------------------------
% Author:
% Jonas Sjöberg     <tel12jsg@student.hig.se>
%
% License:
% Creative Commons Attribution-NonCommercial-ShareAlike 4.0 International
% See LICENSE.md for full licensing information.
% ==============================================================================

\section{Planering}
% TODO: Kort och orienterande om hur ni tänkte genomföra uppgiften.
%       Orienterande om Planering och Genomförande.


% ~~~~~~~~~~~~~~~~~~~~~~~~~~~~~~~~~~~~~~~~~~~~~~~~~~~~~~~~~~~~~~~~~~~~~~~~~~~~~~
\subsection{Arbetsmetod}
% Hur kommer ni att arbeta?  Detta är en lite längre text än den rent
% orienterande texten i Planering och genomförande ovan.

Nedan följer en preliminär redogörelse för den experimentuppställning som används
under laborationen:

\begin{itemize}
  \item Laborationen utförs på en \texttt{ProBook-6545b} laptop som kör
        \texttt{Xubuntu 15.10} på \texttt{Linux 3.19.0-28-generic}.

  \item Rapporten skrivs i \LaTeX\ med texteditorn \texttt{Vim} och kompileras
        till pdf med \texttt{latexmk}.

  \item För revisionskontroll används \texttt{Git}.

  \item Virtualisering sker med \texttt{Oracle VirtualBox} version
        \texttt{5.0.10\_Ubuntu r104061}.

\end{itemize}


% ~~~~~~~~~~~~~~~~~~~~~~~~~~~~~~~~~~~~~~~~~~~~~~~~~~~~~~~~~~~~~~~~~~~~~~~~~~~~~~
\section{Genomförande}
% Här skriver ni vilka steg ni gjorde och resultatet av dem. Ni skall ha med
% information så att vi kan se hur ni har gjort, dvs beskrivande text,
% skärmdumpar, bilder etc.  Längre skärmdumpar, innehåll i relevanta filer och
% större bilder lägger ni i bilagor, som bilaga I, så att de inte tar över en
% sida själva.
% Kommandon som ni skriver i ett skal skall skrivas i detta format, som är
% teckenformatmall ”Exempel” i OpenOffice/LibreOffice. Detta så att de
% skiljer sig från övriga brödtext i stycket.  Detta underlättar läsningen för
% andra, som oss lärare.


% ______________________________________________________________________________
\subsection{Skapande av en virtuell maskin}
Det första steget var att ladda hem en bootbar \texttt{.iso}-fil av
distributionen Debian. Det finns många olika varianter att välja bland och då
jag sedan tidigare är van vid att använda \texttt{xfce} så valde jag att ladda
hem \texttt{debian-live-8.2.0-i386-xfce-desktop} som är en variant av Debian
som levereras med den grafiska miljön \texttt{xfce}. Den Grafiska miljön är
bara en komponent av hela systemet som lätt kan bytas ut vid ett senare
tillfälle, så att välja precis ``rätt'' paktererad version av en distribution
är inte nödvändigt.  Men en färdigpaketerad version brukar kunna förenkla
installationsprocessen.

\screenshot{img/A_new-01}
           {Skärmdump av värdsystemet som kör \texttt{VirtualBox} där en ny
            virtuell maskin skapas genom att klicka den blå ikonen
            \texttt{New}.}
           {fig:A_new-01}

Jag väljer att skapa min miljö som en virtuell maskin. Då jag redan använder
och är bekant med \texttt{VirtualBox} så väljer jag att använda det för att
skapa min virtuella maskin enligt Figur~\ref{fig:A_new-01}.

\screenshot{img/A_new-02}
           {I dialogrutan för en ny virtuell maskin väljs namnet
            \texttt{debian}. Autodetektering baserat på det inmatade
            namnet markerar automatiskt \texttt{Type} samt \texttt{Version}
            som i det här fallet stämmer och är lämpliga inställningar.}
           {}

\screenshot{img/A_new-03}
           {Storleken på RAM-minnet sätts till en lämplig kompromiss mellan
            vad värdsystemet kan tänkas behöva och vad gästsystemet kräver.}
           {}

\screenshot{img/A_new-04}
           {I dialogrutan för hårddisk väljer vi att skapa en ny virtuell
            hårddisk.}
           {}

\screenshot{img/A_new-05}
           {Standardvalet \texttt{VDI} fungerar bra för det här
            användningsområdet. Ytterigare information de olika alternativen
            finns i \cite{virtualbox:vdidetails}.}
            % TODO: FIXA OVANSTÅENDE!
           {}

\screenshot{img/A_new-06}
           {Lagringsutrymmets storlek väljs att växa dynamiskt efter behov.}
           {}

\screenshot{img/A_new-07}
           {Namnet och storleken på disken sätts till \texttt{debian} (vilket
            ger ett filnamn \texttt{debian.vdi} i katalogen för den virtuella
            maskinen) och storleken sätts godtyckligt till \SI{10}{\mega\byte}.}
           {}

\screenshot{img/A_new-08}
           {Ytterligare inställningsmöjligheter kan göras genom att högerklicka
            på den virtuella maskinen vi skapat och välja \texttt{Settings}.
            Under fliken \texttt{Basic} så ser vi att \texttt{Name}, \texttt{Type}
            och \texttt{Version} är valda sedan tidigare.}
           {}

\screenshot{img/A_new-09}
           {Under fliken \texttt{Advanced} aktiveras stöd för delad clipboard,
            vilket gör det möjligt att kopiera och klistra in text mellan
            värdsystemet och den virtuella maskinen.}
           {}

\screenshot{img/A_new-10}
           {Bland alternativen till höger väljs \texttt{Display}.
            Grafikminnet för gästen ökas till värdsystemets högsta möjliga
            \SI{128}{\mega\byte}. Stöd för hårdvaruacceleration finns ej med
            den här kombinationen av grafikkort och version av \texttt{Xorg}.}
           {}

\screenshot{img/A_new-11}
           {Under alternativet \texttt{Storage} finns inställningar för den
            virtuella maskinens lagringsenheter. Under \texttt{Controller: IDE}
            finns en emulerad optisk enhet tillgänglig.
            Och under \texttt{Controller: SATA} syns den virtuella hårddisken
            \texttt{debian.vdi}.}
           {}

\screenshot{img/A_new-12}
           {Diskavbildningsfilen \texttt{debian-live-8.2.0-i386-xfce-desktop} monteras
            genom att klicka på den lilla ikonen av en CD-skiva till höger om
            \texttt{Optical Drive: IDE Secondary Master}. Om kryssrutan
            \texttt{Live CD/DVD} markeras så matas det virtuella mediet ut vid omstart
            efter att installationen är slutförd. Utmatning av media sköts manuellt.}
           {}

\screenshot{img/A_new-13}
           {}
           {}

\screenshot{img/A_new-14}
           {}
           {}

% ______________________________________________________________________________
\subsection{Installation av operativsystemet}
Den nya virtuella maskinen har nu konfigurerats för att fungera
tillfredsställande på värdsystemet och är redo för installation av
operativsystemet.

\screenshot{img/B_install-01}
           {}
           {}

\screenshot{img/B_install-02}
           {}
           {}

\screenshot{img/B_install-03}
           {}
           {}

\screenshot{img/B_install-04}
           {}
           {}

\screenshot{img/B_install-05}
           {}
           {}

\screenshot{img/B_install-06}
           {}
           {}

\screenshot{img/B_install-07}
           {}
           {}

\screenshot{img/B_install-08}
           {}
           {}

\screenshot{img/B_install-09}
           {}
           {}

\screenshot{img/B_install-10}
           {}
           {}

\screenshot{img/B_install-11}
           {}
           {}

\screenshot{img/B_install-12}
           {}
           {}

\screenshot{img/B_install-13}
           {}
           {}

\screenshot{img/B_install-13}
           {}
           {}

\screenshot{img/B_install-14}
           {}
           {}

\screenshot{img/B_install-15}
           {}
           {}

\screenshot{img/B_install-16}
           {}
           {}

\screenshot{img/B_install-17}
           {}
           {}

\screenshot{img/B_install-18}
           {}
           {}

\screenshot{img/B_install-19}
           {}
           {}

\screenshot{img/B_install-20}
           {}
           {}

\screenshot{img/B_install-21}
           {}
           {}

\screenshot{img/B_install-22}
           {}
           {}

\screenshot{img/B_install-23}
           {}
           {}

\screenshot{img/B_install-24}
           {}
           {}

\screenshot{img/B_install-25}
           {}
           {}

\screenshot{img/B_install-26}
           {}
           {}

\screenshot{img/B_install-27}
           {}
           {}

\screenshot{img/B_install-28}
           {}
           {}

\screenshot{img/B_install-29}
           {}
           {}

\screenshot{img/B_install-30}
           {}
           {}

\screenshot{img/B_install-31}
           {}
           {}


% ______________________________________________________________________________
\subsection{Felsökning}

\screenshot{img/C_debug-01}
           {}
           {}

\screenshot{img/C_debug-02}
           {}
           {}

\screenshot{img/C_debug-03}
           {}
           {}


% ______________________________________________________________________________
\subsection{Första start av systemet}

\screenshot{img/D_first-boot-01}
           {}
           {}

\screenshot{img/D_first-boot-02}
           {}
           {}

\screenshot{img/D_first-boot-03}
           {}
           {}


% ______________________________________________________________________________
\subsection{Installation av \texttt{Guest additions}}

\screenshot{img/E_guest-install-01}
           {}
           {}

\screenshot{img/E_guest-install-02}
           {}
           {}

\screenshot{img/E_guest-install-03}
           {}
           {}

\screenshot{img/E_guest-install-04}
           {}
           {}


% ______________________________________________________________________________
\subsection{Felsökning av \texttt{Guest additions}}

\screenshot{img/F_guest-debug-01}
           {}
           {}

\screenshot{img/F_guest-debug-02}
           {}
           {}

\screenshot{img/F_guest-debug-03}
           {}
           {}

\screenshot{img/F_guest-debug-04}
           {}
           {}

\screenshot{img/F_guest-debug-05}
           {}
           {}

\screenshot{img/F_guest-debug-06}
           {}
           {}

\screenshot{img/F_guest-debug-07}
           {}
           {}

\screenshot{img/F_guest-debug-08}
           {}
           {}

\screenshot{img/F_guest-debug-09}
           {}
           {}

\screenshot{img/F_guest-debug-10}
           {}
           {}


% ______________________________________________________________________________
\subsection{Slutförd installation}

\screenshot{img/G_done-01}
           {Efter att systemet startat om sker åter igen inloggning med
            användaren \texttt{jonas}.}
           {}

\screenshot{img/G_done-02}
           {Med \texttt{Guest Additions} installerat kan den virtuella maskinen
            köras i helskärmsläge genom att välja \texttt{Full-screen Mode} i 
            menyn \texttt{View}.}
           {}

\screenshot{img/G_done-03}
           {Bilden visar det slutgiltiga resultatet. Den virtuella maskinen
            körs i helskärm och fungerar precis som om systemet kördes direkt
            ``mot metallen'', precis som värdsystemet. 
            Med \texttt{Guest Additions} installerat justeras fönstret och 
            skärmupplösningen automatiskt för att passa värdsystemet.}
           {}

